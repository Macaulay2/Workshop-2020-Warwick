\documentclass[10pt]{beamer}

\usetheme[progressbar=frametitle]{metropolis}
\usepackage{appendixnumberbeamer}

\usepackage{booktabs}
\usepackage[scale=2]{ccicons}

\usepackage{pgfplots}
\usepgfplotslibrary{dateplot}

\usepackage{xspace}
\newcommand{\themename}{\textbf{\textsc{metropolis}}\xspace}

%%%%%%%%%%%%%%%%%%%%%%%%%%%%%%%%%%%%%%%%%%%%%%%%%%%%%%%%%%%%%%%%%%%%%%%%%%%%%%%%

\setbeamertemplate{title graphic}{
\vbox to 0pt {
\vspace*{22em}
\inserttitlegraphic
}
\nointerlineskip
}

\title{Graphical models}

\date{M2 Workshop 2020, Warwick}
\author{Algebraic statistics project}



\begin{document}

\maketitle

\begin{frame}{Graphical models}

A \textbf{graphical model} is a statistical model associated to a graph, where the
nodes of the graph represent random variables and the edges of the graph encode relationships between the random variables. 

\medskip

Graphical models are used
in many applications because of their ability to model complex
interactions between several random variables, by specifying interactions using only local information about connectivity between the vertices in a graph.


\medskip
GraphicalModels package: compute conditional independence statements and vanishing ideals for Gaussian and discrete graphical models.

\emph{Example on how to introduce a graph and create gaussian ring.}
\end{frame}

\section{Multi-trek ideals}

\section{Maximum likelihood estimation}

\begin{frame}{Maximum likelihood estimation}

$\mathcal{P}_{\Theta}=\lbrace P_{\theta}:\theta\in\rbrace$ statistical model with parameter space $\Theta\subset\mathbb{R}^d$

$X^{1},X^{2},\dots,X^{n}\sim P_\theta$ independent random vector identically distributed according to some unknown probability distribution $P_\theta\in\mathcal{P}_\theta$.

Goal: find $\theta$ that best describes the data.


The \textbf{maximum likelihood estimator} of $\theta$ is the random variable
$$\hat{\theta}=argmax_{\theta\in\Theta}\log\prod_{i=1}^n p_\theta(X^{(i)})$$

The \textbf{maximum likelihood estimate} is $\theta$ for the data $x^{(1)},\dots,x^{(n)}$ is the realization of $\hat{\theta}$ obtained by the substitution $X^{(1)}=x^{(1)},\dots,X^{(n)}=x^{(n)}$.

\end{frame}

\begin{frame}{GraphicalModelsMLE}
What do we want it to do?

Compute maximum likelihood estimations for discrete and Gaussian graphical models. 

What have we done?

Revisit a package started in a previous workshop and add...numerical tools to solve optimization problems? MLE for undirected graphs?

\end{frame}

\begin{frame}{Example}

Example from Lecture notes on Algebraic Statistics (Drton, Sturmfels, Sullivant)

$$\Theta_2=\lbrace\Sigma\in PD_4: (\Sigma^{-1})_{13}=(\Sigma^{-1})_{24}=0\rbrace$$

$\Theta=R^m\times\Theta_2$ undirected Gaussian graphical model


associated to the cyclic graph with vertex 1,2,3,4 and edges 12, 23, 34, 14 (picture instead of definition?)

Goal: Maximize likelihood equations.

\emph{Compute example using new eigensolver from the numerical group.} 
\end{frame}

\end{document}
