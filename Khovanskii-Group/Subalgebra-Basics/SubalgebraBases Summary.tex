\documentclass[11pt]{amsart}
\usepackage[margin=1in]{geometry}
\usepackage{xcolor}

\title{Summary of the \texttt{subalgebraBases} function}

\DeclareMathOperator{\lead}{lead}

\newcommand{\alert}[1]{{\color{red}#1}}

\begin{document}
\maketitle

The input to the algorithm is
\begin{itemize}
\item an ambient ring,
\item a finite list of subring generators, and
\item a maximum degree
\end{itemize}
The steps to the algorithm are
\begin{enumerate}
\item Take all of the subring generators eliminate any which are 0 and discard those of degree 0 or larger than the limit.  Add the remaining generators to the list of pending.\label{step:discard}
\item Take the lowest degree in pending, do easy reductions with them and replace them.  \alert{This uses the \texttt{rowReduce} command, which I do not fully understand.}\label{step:lowestdegree}
\item With the (new) lowest degree in the pending list, remove those from pending and add these generators to the subalgebra basis.
\item Construct the syzygy ideal\label{step:syzygy}
\begin{itemize}
\item Let $\{p_0,\dots,p_n\}$ be the generators of the ambient ring.
\item Let $\{p_{n+1},\dots,p_m\}$ be the subalgebra generators computed so far.
\item The syzygy ideal is generated by terms of the form $p_j-\lead(p_j)$.  Here, $n<j\leq m$ and $\lead(p_j)$ writes the leading term of $p_j$ in terms of $p_0,\dots,p_n$.  The monomial order for this ideal eliminates $p_0,\dots,p_n$.
\item The importance of this ideal is that $p^\alpha-\lead(p^\alpha)$ is in the ideal where $p^\alpha$ is any product of subalgebra generators.
\item In particular, $p^\alpha-p^\beta$ is in the ideal if and only if $\lead(p^\alpha)=\lead(p^\beta)$, i.e., a syzygy.
\end{itemize}
\item The number of loops start at one more than this processed degree to represent several empty loops.
\item The while loop continues until there have been enough loops completed or the subalgebra basis has been proven to be completed.
\begin{enumerate}
\item Compute the Gr\"obner basis of the syzygy ideal, but only the generators of degree less than the current degree.  This choice prevents us from adding extra work because the syzygy ideal changes in the next loop.  \alert{Could this be improved to avoid extra work?}
\item Only look at syzygy generators without $p_0,\dots,p_n$ that are of the current degree.  \alert{Why is it enough to only look at the current degree?  Have the smaller degrees already been processed somewhere?}
\item If there are pending of the current degree, add them to the mix.
\item Perform subduction on this list of potential generators with respect to the subalgebra basis so far.  \alert{This should give a remainder where no term is divisible by a leading term of the subalgebra basis so far.}
\item If there is anything left, insert the remaining entries into pending and perform Steps (\ref{step:lowestdegree}) through (\ref{step:syzygy}).
\item Otherwise, if there are no more pending, the partial subalgebra basis is done, and there were no generators of even higher degree that were discarded in Step (\ref{step:discard}).  Then, there are no more syzygies possible and the algorithm has found a finite subalgebra basis.
\end{enumerate}
\end{enumerate}

\end{document}